\section{Modelling Secondary Education in PDDL}
This project employs a structured ontology in PDDL, encompassing types, constants, and predicates to model secondary education. Types include courses, course levels, weeks, units, extracurricular activities, grades, preferred learning styles, and student support requirements. Constants consist of strategies, school calendar weeks, units, extracurricular activities, predicted grades, and student preferences and support requirements. Predicates establish relationships such as finished courses, course enrollment, predicted grades, completion of weeks, units, and extracurricular activities, participation in study groups, adoption of learning strategies, identification of support needs, and provision of support.

\subsection{LEARNS Architecture}
Planning learning paths for students requires the initial step of creating a framework, which is one that contains 36 weeks. There are $32$ weeks for learning, and $4$ weeks for students to be assessed by means of unit tests. The successful completion of each week is systematically recorded through the \texttt{done-week} predicate, as denoted in Figures~\ref{fig:do-week-one} and~\ref{fig:do-week-n}. Figure~\ref{fig:do-week-n} is an action that is chained from Figure~\ref{fig:do-week-one}, and is executed for each week that is to be completed (week-n), $n$ is in the range of $2 \leq n \leq 8$ and represents the weeks $2$ through $8$ respectively.

\begin{figure}[t]
    \begin{lstlisting}
    (:action do-week-one
      :parameters (?s - student ?c - course ?l - course-level)
      :precondition (and 
        (takes-course ?s ?c ?l)
        (not(done-week week-one ?s ?c ?l))
      )
      :effect (and 
        (done-week week-one ?s ?c ?l)
      )
    )
    \end{lstlisting}
    \caption{PDDL action that commences study for the student}\label{fig:pddl-action-do-week-one}
\end{figure}
\begin{figure}[t]
    \begin{lstlisting}
    (:action do-week-n
      :parameters (?s - student ?c - course ?l - course-level)
      :precondition (and 
        (not(done-week week-n ?s ?c ?l))
        (done-week week-(n-1) ?s ?c ?l)
      )
      :effect (and 
        (done-week week-n ?s ?c ?l)
      )
    )
    \end{lstlisting}
    \caption{PDDL action that continues study for the student by permitting them to take the $n$th week of study.}\label{fig:do-week-n}
\end{figure}
    

After eight weeks have passed, students are provided with a unit test to complete. The \texttt{done-unit} predicate in Figures~\ref{fig:take-unit-one} and~\ref{fig:take-unit-n} denote their passing of the unit \textemdash~assuming that all students who take a unit will pass it. This is a simplification of the problem, as it is possible for students to fail a unit test. However, this is not a concern for this project, as the focus is on the planning of learning paths, rather than the planning of resits. Following the accomplishment of a unit, students are then tasked with a subsequent 8-week cycle, this cycle persists until the student has completed all four units from the course.

\begin{figure}[t]
    \begin{lstlisting}
    (:action take-unit-one
      :parameters (?s - student ?c - course ?l - course-level)
      :precondition (and 
        (takes-course ?s ?c ?l)
        (not(done-unit unit-one ?s ?c ?l))
        (done-week week-eight ?s ?c ?l)
      )
      :effect (and
        (done-unit unit-one ?s ?c ?l)
        (not(done-week week-eight ?s ?c ?l))
        (not(done-week week-seven ?s ?c ?l))
        (not(done-week week-six ?s ?c ?l))
        (not(done-week week-five ?s ?c ?l))
        (not(done-week week-four ?s ?c ?l))
        (not(done-week week-three ?s ?c ?l))
        (not(done-week week-two ?s ?c ?l))
        (not(done-week week-one ?s ?c ?l))
      )
    )
    \end{lstlisting}
    \caption{PDDL action that allows a student to take unit one}\label{fig:pddl-action-take-unit-one}
\end{figure}
\begin{figure}[t]
  \small  % Adjust font size
  \begin{align*}
  &\pre(\text{take-unit-n}) = \\
  &\quad \text{takes-course}(\text{?s}, \text{?c}, \text{?l}) \land \\
  &\quad \lnot \text{done-unit}(\text{unit-n}, \text{?s}, \text{?c}, \text{?l}) \land \\
  &\quad \text{done-unit}(\text{unit-(n-1)}, \text{?s}, \text{?c}, \text{?l}) \land \\
  &\quad \text{done-week}(\text{week-eight}, \text{?s}, \text{?c}, \text{?l}) \\
  &\eff(\text{take-unit-n}) = \\
  &\quad \text{done-unit}(\text{unit-n}, \text{?s}, \text{?c}, \text{?l}) \land \\
  &\quad \lnot \text{done-week}(\text{week-eight}, \text{?s}, \text{?c}, \text{?l}) \land \\
  &\quad \lnot \text{done-week}(\text{week-seven}, \text{?s}, \text{?c}, \text{?l}) \land \\
  &\quad \lnot \text{done-week}(\text{week-six}, \text{?s}, \text{?c}, \text{?l}) \land \\
  &\quad \lnot \text{done-week}(\text{week-five}, \text{?s}, \text{?c}, \text{?l}) \land \\
  &\quad \lnot \text{done-week}(\text{week-four}, \text{?s}, \text{?c}, \text{?l}) \land \\
  &\quad \lnot \text{done-week}(\text{week-three}, \text{?s}, \text{?c}, \text{?l}) \land \\
  &\quad \lnot \text{done-week}(\text{week-two}, \text{?s}, \text{?c}, \text{?l}) \land \\
  &\quad \lnot \text{done-week}(\text{week-one}, \text{?s}, \text{?c}, \text{?l})
  \end{align*}
  \caption{Logical formalisation of the PDDL action for a student taking the $n$th unit}\label{fig:take-unit-n}
\end{figure}

A student has completed a course when all four units are complete, and is denoted by the \texttt{finished-course} predicate. Figure~\ref{fig:finish-course} provides the action to check that the student has completed the course. This is a prerequisite for the student to be able to take the final exam \textemdash~of which is not monitored by this system, as it is an application of automated planning to generate learning paths for students with additional support requirements, not for handling the logistics of the school.

\begin{figure}[t]
  \small  % Adjust font size
  \begin{align*}
  &\pre(\text{finish-course}) = \\
  &\quad \text{done-unit}(\text{unit-four}, \text{?s}, \text{?c}, \text{?l}) \land \\
  &\quad \text{takes-course}(\text{?s}, \text{?c}, \text{?l}) \\
  &\eff(\text{finish-course}) = \\
  &\quad \text{finished-course}(\text{?s}, \text{?c}, \text{?l})
  \end{align*}
  \caption{Logical formalisation of the PDDL action for finishing a course}\label{fig:finish-course}
\end{figure}

Alongside their academics, extracurricular activities play a vital role in student development. For each unit, students can participate in two extracurricular activities. Grade-a students are expected to commit to both, while grade-b students typically commit to one. Figures~\ref{fig:extra-curricular-one} and~\ref{fig:extra-curricular-n} illustrate this academic involvement, highlighting the sequential nature of these activities.

\begin{figure}[t]
    \begin{lstlisting}
    (:action do-extra-curricular-one
      :parameters (?s - student ?c - course ?l - course-level)
      :precondition (and 
        (takes-course ?s ?c ?l)
        (or
          (grade a ?s ?c ?l)
          (grade b ?s ?c ?l)
        )
        (not
          (done-extra-curricular ec-one ?s ?c ?l)
        )
      )
      :effect (and 
        (done-extra-curricular ec-one ?s ?c ?l)
      )
    )
    \end{lstlisting}
    \caption{PDDL action to allow a student to take the first extracurricular activity}\label{fig:extra-curricular-one}
\end{figure}
\begin{figure}[t]
    \begin{lstlisting}
    (:action do-extra-curricular-n
      :parameters (?s - student ?c - course ?l - course-level)
      :precondition (and 
        (takes-course ?s ?c ?l)
        (grade a ?s ?c ?l)
        (not
          (done-extra-curricular ec-n ?s ?c ?l)
        )
        (done-extra-curricular ec-(n-1) ?s ?c ?l)
      )
      :effect (and 
        (done-extra-curricular ec-n ?s ?c ?l)
      )
    )
    \end{lstlisting}
    \caption{PDDL action that has a student to the $n$th extracurricular activity}\label{fig:extra-curricular-n}
\end{figure}

Extracurricular activities alternate, starting with \texttt{do-extra-curricular-one} and progressing to (roughly) \texttt{do-extra-curricular-n}. This pattern repeats for each unit, and completion is indicated by the \texttt{done-extra-curricular} predicate. The approximation stems from the earlier statement that grade-a or grade-b students may engage in either one or two extracurricular activities. However, this simplification overlooks the possibility of non-grade-a or non-grade-b students participating and grade-a or grade-b students deviating from the specified number of extracurricular engagements.

Students willing to engage in collaborative group dynamics can participate in group-oriented extracurricular activities. This collaboration is depicted in Figures~\ref{fig:team-extra-curricular-one} and~\ref{fig:team-extra-curricular-n}, illustrating the synergistic environment fostered through teamwork in extracurricular pursuits.

\begin{figure}[t]
  \small  % Adjust font size
  \begin{align*}
  &\pre(\text{do-team-extra-curricular-one}) = \\
  &\quad \text{takes-course}(\text{?s}, \text{?c}, \text{?l}) \land \\
  &\quad (\text{grade a}(\text{?s}, \text{?c}, \text{?l}) \lor \\
  &\quad \text{grade b}(\text{?s}, \text{?c}, \text{?l})) \land \\
  &\quad \text{study-group}(\text{?c}, \text{?l}) \land \\
  &\quad \text{uses-strategy}(\text{?s}, \text{teamwork}) \land \\
  &\quad \lnot \text{done-extra-curricular}(\text{ec-one}, \text{?s}, \text{?c}, \text{?l}) \\
  &\eff(\text{do-team-extra-curricular-one}) = \\
  &\quad \text{done-extra-curricular}(\text{ec-one}, \text{?s}, \text{?c}, \text{?l})
  \end{align*}
  \caption{Logical formalisation of the PDDL action for a student doing the first extracurricular activity as part of a team}\label{fig:team-extra-curricular-one}
\end{figure}
\begin{figure}[t]
    \begin{lstlisting}
    (:action do-team-extra-curricular-n
      :parameters (?s - student ?c - course ?l - course-level)
      :precondition (and 
        (takes-course ?s ?c ?l)
        (grade a ?s ?c ?l)
        (not
          (done-extra-curricular ec-n ?s ?c ?l)
        )
        (study-group ?c ?l)
        (uses-strategy ?s teamwork)
        (done-extra-curricular ec-(n-1) ?s ?c ?l)
      )
      :effect (and 
        (done-extra-curricular ec-n ?s ?c ?l)
      )
    )
    \end{lstlisting}
    \caption{PDDL action to allow a student to do the $n$th extracurricular activity as part of a team.}\label{fig:team-extra-curricular-n}
    \end{figure}

\subsection{Support Structures}
\begin{figure}[t]
    \begin{lstlisting}
    (:action recommend-improving-communications-workshop
      :parameters (?s - student)
      :precondition (and
        (or
          (has-support-need ?s asc-asd)
          (has-support-need ?s language)
          (uses-strategy ?s teamwork)
          (uses-strategy ?s project-based)
          (uses-strategy ?s blended-learning)
        )
      )
      :effect (and 
        (given-support ?s improving-comms-workshop)
      )
    )
    \end{lstlisting}
    \caption{PDDL action to recommend a student attends a workshop on improving their communication skills.}\label{fig:improve-comms-workshop}
\end{figure}
\begin{figure}[t]
    \begin{lstlisting}
    (:action recommend-reading-group
      :parameters (?s - student)
      :precondition (and 
        (has-support-need ?s language)
        (has-support-need ?s blind-visual)
      )
      :effect (and 
        (given-support ?s reading-group)
      )
    )
    \end{lstlisting}
    \caption{PDDL action to recommend a student attends a reading group.}\label{fig:pddl-action-reading-group}
\end{figure}
\begin{figure}[t]
    \begin{lstlisting}
    (:action recommend-technological-assistance
      :parameters (?s - student)
      :precondition (and 
        (or
          (has-support-need ?s asc-asd)
          (has-support-need ?s language)
          (has-support-need ?s deaf-hearing)
          (has-support-need ?s blind-visual)
          (uses-strategy ?s technological-tools)
          (uses-strategy ?s blended-learning)
        )
      )
      :effect (and 
        (given-support ?s tech-assist)
      )
    )
    \end{lstlisting}
    \caption{PDDL action to recommend a student receives a digitised format to assist with their learning.}\label{fig:tech-assist}
\end{figure}
\begin{figure}[t]
    \begin{lstlisting}
    (:action recommend-pomodoro-study
      :parameters (?s - student)
      :precondition (and 
        (or
          (has-support-need ?s asc-asd)
          (has-support-need ?s tourettes)
          (uses-strategy ?s pomodoro)
          (uses-strategy ?s teamwork)
        )
      )
      :effect (and 
        (given-support ?s pomo)
      )
    )
    \end{lstlisting}
    \caption{PDDL Action: recommend-pomodoro-study}\label{fig:pomo-study}
    \end{figure}
\begin{figure}[t]
    \begin{lstlisting}
    (:action recommend-gamify-learning
      :parameters (?s - student)
      :precondition (and
        (or
          (has-support-need ?s asc-asd)
          (has-support-need ?s tourettes)
          (uses-strategy ?s gamification)
          (uses-strategy ?s blended-learning)
        )
      )
      :effect (and 
        (given-support ?s gamify-learning)
      )
    )
    \end{lstlisting}
    \caption{PDDL action to recommend a student's learning is gamified.}\label{fig:gamify-learning}
\end{figure}
\begin{figure}[t]
    \begin{lstlisting}
    (:action recommend-isolated-study-where-appropriate
      :parameters (?s - student)
      :precondition (and 
        (or
          (has-support-need ?s tourettes)
          (has-support-need ?s asc-asd)
          (has-support-need ?s social-other)
        )
      )
      :effect (and 
        (given-support ?s isolated)
      )
    )
    \end{lstlisting}
    \caption{PDDL action to recommend students who may benefit with an option for isolated study}\label{fig:pddl-action-isolated-study}
\end{figure}

\begin{itemize}
    \item We recognise through our independent research that if a student is struggling with their communication skills, it may be possible to provide a workshop to help them improve. This could be for students who have social issues, are on the autism spectrum or may have linguistic issues. Additionally, a student could also prefer to use specific learning styles that require aid of a communications improvement workshop \textemdash~such as teamwork, project-based learning or blended learning for those that are unable to be physically in school. See Figure~\ref{fig:improve-comms-workshop}.
    \item Should a student face particular difficulty with their reading ability, or may benefit from using tools such as audiobooks, we can provide a reading group workshop to help them with that. This is detailed in Figure~\ref{fig:reading-group}.
    \item If a student prefers to learn with the assistance of technology, we can provide them with a tool to help them with their learning, which could be for a variety of reasons. These reasons could include the student being on the autism spectrum, may have difficulties with language, could be profoundly deaf or hard of hearing, could be fully blind or visually impaired, or they may prefer to just use technology to assist their learning or be learning from a distance. This is detailed in Figure~\ref{fig:tech-assist}.
    \item Additionally, students may prefer and/or benefit from breaking up their study time into smaller chunks, which is known as the Pomodoro technique. This is detailed in Figure~\ref{fig:pomo-study}.
\end{itemize}