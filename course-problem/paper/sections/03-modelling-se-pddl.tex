\section{Modelling Secondary Education in PDDL}
This project utilises a structured ontology comprising types, constants, and predicates to frame the model of secondary education in PDDL\@. The defined types encompass courses, course levels, weeks, units, extracurricular activities, grades, preferred learning styles, and various support requirements that students may have. Our constants include strategies such as the pomodoro study technique, the weeks in the school calendar, units, extracurricular activities, predicted grades, and the student's perceived support preferences, and the student's support requirements. Lastly, the predicates capture relationships such as finished courses, course enrollment, predicted grades, completion of weeks, units, and extracurricular activities, participation in study groups, adoption of specific learning strategies, identification of support needs, and provision of support.
\begin{enumerate}
    \item \textbf{Technical description of the LEARNS architecture.}
    \item \textbf{Application of PDDL for modelling various scenarios.}
    \item \textbf{Presentation of structures used in LEARNS for learning style recognition.}
\end{enumerate}

\subsection{LEARNS Architecture}

\begin{figure}[t]
    \begin{lstlisting}
    (:action recommend-improving-communications-workshop
      :parameters (?s - student)
      :precondition (and
        (or
          (has-support-need ?s asc-asd)
          (has-support-need ?s language)
          (uses-strategy ?s teamwork)
          (uses-strategy ?s project-based)
          (uses-strategy ?s blended-learning)
        )
      )
      :effect (and 
        (given-support ?s improving-comms-workshop)
      )
    )
    \end{lstlisting}
    \caption{PDDL action to recommend a student attends a workshop on improving their communication skills.}\label{fig:improve-comms-workshop}
\end{figure}
\begin{figure}[t]
    \begin{lstlisting}
    (:action recommend-reading-group
      :parameters (?s - student)
      :precondition (and 
        (has-support-need ?s language)
        (has-support-need ?s blind-visual)
      )
      :effect (and 
        (given-support ?s reading-group)
      )
    )
    \end{lstlisting}
    \caption{PDDL action to recommend a student attends a reading group.}\label{fig:pddl-action-reading-group}
\end{figure}
\begin{figure}[t]
    \begin{lstlisting}
    (:action recommend-technological-assistance
      :parameters (?s - student)
      :precondition (and 
        (or
          (has-support-need ?s asc-asd)
          (has-support-need ?s language)
          (has-support-need ?s deaf-hearing)
          (has-support-need ?s blind-visual)
          (uses-strategy ?s technological-tools)
          (uses-strategy ?s blended-learning)
        )
      )
      :effect (and 
        (given-support ?s tech-assist)
      )
    )
    \end{lstlisting}
    \caption{PDDL action to recommend a student receives a digitised format to assist with their learning.}\label{fig:tech-assist}
\end{figure}
\begin{figure}[t]
    \begin{lstlisting}
    (:action recommend-pomodoro-study
      :parameters (?s - student)
      :precondition (and 
        (or
          (has-support-need ?s asc-asd)
          (has-support-need ?s tourettes)
          (uses-strategy ?s pomodoro)
          (uses-strategy ?s teamwork)
        )
      )
      :effect (and 
        (given-support ?s pomo)
      )
    )
    \end{lstlisting}
    \caption{PDDL Action: recommend-pomodoro-study}\label{fig:pomo-study}
    \end{figure}
\begin{figure}[t]
    \begin{lstlisting}
    (:action recommend-gamify-learning
      :parameters (?s - student)
      :precondition (and
        (or
          (has-support-need ?s asc-asd)
          (has-support-need ?s tourettes)
          (uses-strategy ?s gamification)
          (uses-strategy ?s blended-learning)
        )
      )
      :effect (and 
        (given-support ?s gamify-learning)
      )
    )
    \end{lstlisting}
    \caption{PDDL action to recommend a student's learning is gamified.}\label{fig:gamify-learning}
\end{figure}
\begin{figure}[t]
    \begin{lstlisting}
    (:action recommend-isolated-study-where-appropriate
      :parameters (?s - student)
      :precondition (and 
        (or
          (has-support-need ?s tourettes)
          (has-support-need ?s asc-asd)
          (has-support-need ?s social-other)
        )
      )
      :effect (and 
        (given-support ?s isolated)
      )
    )
    \end{lstlisting}
    \caption{PDDL action to recommend students who may benefit with an option for isolated study}\label{fig:pddl-action-isolated-study}
\end{figure}

\begin{itemize}
    \item Begin by defining a series of weeks in a predicate called \texttt{done-week~?w~?s~?c~?l}.
    \item After completing eight weeks, a student can take a unit.
    \item Upon finishing a unit, the student is expected to do another 8 weeks, rinse and repeat.
    \item After finishing all four units, the student has finished the course.
    \item For every unit being taken, there are two extracurricular activities for the students to take, if the student is a grade-a student then they take both, otherwise if they are a grade b student they will take only one. See Figures~\ref{fig:extra-curricular-one} and~\ref{fig:extra-curricular-n}.
    \item 
\end{itemize}

\begin{figure}[t]
    \begin{lstlisting}
    (:action do-extra-curricular-one
      :parameters (?s - student ?c - course ?l - course-level)
      :precondition (and 
        (takes-course ?s ?c ?l)
        (or
          (grade a ?s ?c ?l)
          (grade b ?s ?c ?l)
        )
        (not
          (done-extra-curricular ec-one ?s ?c ?l)
        )
      )
      :effect (and 
        (done-extra-curricular ec-one ?s ?c ?l)
      )
    )
    \end{lstlisting}
    \caption{PDDL action to allow a student to take the first extracurricular activity}\label{fig:extra-curricular-one}
\end{figure}
\begin{figure}[t]
    \begin{lstlisting}
    (:action do-extra-curricular-n
      :parameters (?s - student ?c - course ?l - course-level)
      :precondition (and 
        (takes-course ?s ?c ?l)
        (grade a ?s ?c ?l)
        (not
          (done-extra-curricular ec-n ?s ?c ?l)
        )
        (done-extra-curricular ec-(n-1) ?s ?c ?l)
      )
      :effect (and 
        (done-extra-curricular ec-n ?s ?c ?l)
      )
    )
    \end{lstlisting}
    \caption{PDDL action that has a student to the $n$th extracurricular activity}\label{fig:extra-curricular-n}
\end{figure}
\begin{figure}[t]
  \small  % Adjust font size
  \begin{align*}
  &\pre(\text{do-team-extra-curricular-one}) = \\
  &\quad \text{takes-course}(\text{?s}, \text{?c}, \text{?l}) \land \\
  &\quad (\text{grade a}(\text{?s}, \text{?c}, \text{?l}) \lor \\
  &\quad \text{grade b}(\text{?s}, \text{?c}, \text{?l})) \land \\
  &\quad \text{study-group}(\text{?c}, \text{?l}) \land \\
  &\quad \text{uses-strategy}(\text{?s}, \text{teamwork}) \land \\
  &\quad \lnot \text{done-extra-curricular}(\text{ec-one}, \text{?s}, \text{?c}, \text{?l}) \\
  &\eff(\text{do-team-extra-curricular-one}) = \\
  &\quad \text{done-extra-curricular}(\text{ec-one}, \text{?s}, \text{?c}, \text{?l})
  \end{align*}
  \caption{Logical formalisation of the PDDL action for a student doing the first extracurricular activity as part of a team}\label{fig:team-extra-curricular-one}
\end{figure}
\begin{figure}[t]
    \begin{lstlisting}
    (:action do-team-extra-curricular-n
      :parameters (?s - student ?c - course ?l - course-level)
      :precondition (and 
        (takes-course ?s ?c ?l)
        (grade a ?s ?c ?l)
        (not
          (done-extra-curricular ec-n ?s ?c ?l)
        )
        (study-group ?c ?l)
        (uses-strategy ?s teamwork)
        (done-extra-curricular ec-(n-1) ?s ?c ?l)
      )
      :effect (and 
        (done-extra-curricular ec-n ?s ?c ?l)
      )
    )
    \end{lstlisting}
    \caption{PDDL action to allow a student to do the $n$th extracurricular activity as part of a team.}\label{fig:team-extra-curricular-n}
    \end{figure}
\begin{figure}[t]
    \begin{lstlisting}
    (:action take-unit-one
      :parameters (?s - student ?c - course ?l - course-level)
      :precondition (and 
        (takes-course ?s ?c ?l)
        (not(done-unit unit-one ?s ?c ?l))
        (done-week week-eight ?s ?c ?l)
      )
      :effect (and
        (done-unit unit-one ?s ?c ?l)
        (not(done-week week-eight ?s ?c ?l))
        (not(done-week week-seven ?s ?c ?l))
        (not(done-week week-six ?s ?c ?l))
        (not(done-week week-five ?s ?c ?l))
        (not(done-week week-four ?s ?c ?l))
        (not(done-week week-three ?s ?c ?l))
        (not(done-week week-two ?s ?c ?l))
        (not(done-week week-one ?s ?c ?l))
      )
    )
    \end{lstlisting}
    \caption{PDDL action that allows a student to take unit one}\label{fig:pddl-action-take-unit-one}
\end{figure}
\begin{figure}[t]
  \small  % Adjust font size
  \begin{align*}
  &\pre(\text{take-unit-n}) = \\
  &\quad \text{takes-course}(\text{?s}, \text{?c}, \text{?l}) \land \\
  &\quad \lnot \text{done-unit}(\text{unit-n}, \text{?s}, \text{?c}, \text{?l}) \land \\
  &\quad \text{done-unit}(\text{unit-(n-1)}, \text{?s}, \text{?c}, \text{?l}) \land \\
  &\quad \text{done-week}(\text{week-eight}, \text{?s}, \text{?c}, \text{?l}) \\
  &\eff(\text{take-unit-n}) = \\
  &\quad \text{done-unit}(\text{unit-n}, \text{?s}, \text{?c}, \text{?l}) \land \\
  &\quad \lnot \text{done-week}(\text{week-eight}, \text{?s}, \text{?c}, \text{?l}) \land \\
  &\quad \lnot \text{done-week}(\text{week-seven}, \text{?s}, \text{?c}, \text{?l}) \land \\
  &\quad \lnot \text{done-week}(\text{week-six}, \text{?s}, \text{?c}, \text{?l}) \land \\
  &\quad \lnot \text{done-week}(\text{week-five}, \text{?s}, \text{?c}, \text{?l}) \land \\
  &\quad \lnot \text{done-week}(\text{week-four}, \text{?s}, \text{?c}, \text{?l}) \land \\
  &\quad \lnot \text{done-week}(\text{week-three}, \text{?s}, \text{?c}, \text{?l}) \land \\
  &\quad \lnot \text{done-week}(\text{week-two}, \text{?s}, \text{?c}, \text{?l}) \land \\
  &\quad \lnot \text{done-week}(\text{week-one}, \text{?s}, \text{?c}, \text{?l})
  \end{align*}
  \caption{Logical formalisation of the PDDL action for a student taking the $n$th unit}\label{fig:take-unit-n}
\end{figure}
\begin{figure}[t]
    \begin{lstlisting}
    (:action do-week-one
      :parameters (?s - student ?c - course ?l - course-level)
      :precondition (and 
        (takes-course ?s ?c ?l)
        (not(done-week week-one ?s ?c ?l))
      )
      :effect (and 
        (done-week week-one ?s ?c ?l)
      )
    )
    \end{lstlisting}
    \caption{PDDL action that commences study for the student}\label{fig:pddl-action-do-week-one}
\end{figure}
\begin{figure}[t]
    \begin{lstlisting}
    (:action do-week-n
      :parameters (?s - student ?c - course ?l - course-level)
      :precondition (and 
        (not(done-week week-n ?s ?c ?l))
        (done-week week-(n-1) ?s ?c ?l)
      )
      :effect (and 
        (done-week week-n ?s ?c ?l)
      )
    )
    \end{lstlisting}
    \caption{PDDL action that continues study for the student by permitting them to take the $n$th week of study.}\label{fig:do-week-n}
\end{figure}
    