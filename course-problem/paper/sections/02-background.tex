\section{Background}
\citet{Castillo2009} introduces an integrated approach for automatically extracting planning domains from a well-structured Learning Object Repository (LOR) by utilising standard metadata labelling within objects. The method is integrated into popular e-learning platforms, ILIAS and Moodle, marking a significant initial step towards automated deployment of Hierarchical Task Network (HTN) planners~\cite{georgievski2014overview} for generating customised learning paths tailored to individual student needs. The technical advantages include the efficient use of an AI-driven HTN planner, circumventing the manual and tedious process of encoding learning designs. This advancement holds promise for the broader implementation of e-learning technologies.

From the instructor's perspective, the approach offers several notable benefits. It simplifies the painstaking task of crafting adapted learning designs for diverse student sets, acting as an enabling technology for standards adoption. The streamlined design process for learner-centric paths enhances instructor productivity, allowing a focus on creating more robust Learning Object Repositories (LORs) and courses. The reliance on comprehensive metadata ensures that the quality and adaptation of the plans align with those crafted manually, though empirical validation of this assumption remains a subject for future investigation.

However, the success of their approach is contingent on factors such as the number and redundancy of defined learning objects and the quality of their labelling by the instructor. Beyond the scope of the approach are considerations related to the initial definition and labelling of learning objects, falling within the broader domain of instructors adhering to the Learner-Centred Design paradigm. Several technical issues merit further exploration, including the adaptation of learning paths to runtime information and the comparison of non-HTN planners for performance and expressiveness, with potential applications in domains such as workflow composition.

\citet{Ouyang2023} for a future focus on integrating AI into learning analytics, aiming to employ AI for data organisation, analysis, and informed decision-making to support student success, as depicted by~\citet{Pelletier2022}. Specifically, the study addresses the gap in existing AI prediction models, which predominantly concentrate on model development and optimisation rather than leveraging AI for timely and continuous feedback to enhance student learning quality. To fill this void,~\citet{Ouyang2023} integrates an AI performance prediction model with learning analytics visualisation and feedback, conducting quasi-experimental research in an online engineering course. The empirical results demonstrate competitive advantages of the integrated approach, showcasing improvements in students' engagement, performances, and learning perceptions. The paper suggests future work should expand across educational contexts, course subjects, and sample sizes to validate the empirical findings. Overall, the research by~\citet{Ouyang2023} contributes by proposing an integrated approach, conducting empirical studies, and offering implications for advancing AI-driven learning analytics to bridge the gap between AI model development and educational application.

\subsubsection{Challenges Specific to Secondary Education} 
The challenges specific to secondary education that demand innovative solutions include accommodating a diverse range of learning styles and integrating technology into learning solutions. This is exemplified by~\citet{Szymkowiak2021} findings (2021), which highlight the inadequacy of traditional education methods for generation Z. This generation exhibits a clear preference for incorporating modern technology into their learning processes. Addressing this, facilitating adaptable learning through platforms such as games, mobile apps, videos, and podcasts becomes crucial.~\citet{Szymkowiak2021} underscores the significance of providing these resources and incorporating technology into the classroom, emphasising its marked improvement in the educational experience for generation Z students. Recognising their shorter attention spans, impatience, and reliance on digital media, educators can adapt to these preferences, as evidenced by insights gleaned from a questionnaire administered on a prominent peer-to-peer learning platform during the active pursuit of educational insights.

\subsubsection{Adaptive Learning and Intelligent Tutoring Systems} Adaptive learning systems, employing AI algorithms, furnish tailored educational experiences by adjusting content and activities based on individual learners' needs.~\citet{ElSabagh2021} highlights the creation of a unique learning path for each student through real-time performance data as a key aspect of personalisation. 

Immediate feedback constitutes a crucial feature, empowering students to promptly identify and rectify errors, thereby cultivating a deeper comprehension of the material.~\citet{munoz2022systematic} underscores the imperative for heightened rigour and diversity in adaptive learning research, observing a scarcity of studies and advocating for a focus on adapting to contextual variations. Future research efforts are recommended for meta-analytic purposes to consolidate findings and augment understanding.

Furthermore, adaptive learning systems yield data-driven insights by collecting and analysing data on student interactions.~\citet{Apoki2022} evaluation of pedagogical agents in personalised adaptive learning systems accentuates their adaptive and intelligent roles. The study addresses challenges in online settings, offering insights into the potential impact of incorporating pedagogical agents, including improvements in performance, task completion, increased motivation, and heightened engagement.

There exists a potential drawback in the form of an overemphasis on data and algorithms, however. This inclination towards a data-driven approach may inadvertently lead to a mechanistic perspective on education, neglecting the crucial human element. The work by~\citet{Rojas2022} extensively delves into collaborative problem-solving and real-time feedback but falls short of providing concrete applications within a classroom context.

The research conducted by~\citet{Mirata2020} highlights another significant limitation related to resource requirements. Despite the theoretical advantages of adaptive learning, its widespread adoption in universities faces challenges. The study emphasises the need for further investigation into issues surrounding institutional support, requisite skills, and the justification of investments. To enhance the adoption of adaptive learning, the paper advocates for additional research employing diverse case studies. These case studies should explore general trends under varying conditions, offering a comprehensive framework for organising and summarising adaptive learning research.

In conclusion, despite the challenges outlined in these studies, the potential benefits of adaptive learning, such as catering to diverse student needs and enhancing education quality, underscore the necessity for extensive research. This research is crucial for fostering increased adoption, with the prospect of yielding substantial returns for students, educational institutions, and the broader regional economy.

\subsubsection{Intelligent Tutoring Systems}Intelligent Tutoring Systems (ITS) emulate one-on-one tutoring experiences by employing AI planning techniques to offer real-time guidance and adapt instructional approaches. Key features encompass the provision of immediate feedback, clarifications, and guidance to students during problem-solving, as elucidated by~\citet{Corbett1997}. In subjects requiring sequential problem-solving, ITS provides step-by-step assistance, aiding students in grasping complex concepts, as discussed by~\citet{burns2013intelligent}. The system continually assesses a student's knowledge and skills, tailoring practice exercises and content based on individual performance.

Despite the promising aspects of ITS, notable limitations deserve consideration. Firstly, ITS may exhibit a lack of contextual understanding, potentially leading to technically correct but pedagogically inappropriate recommendations, as observed in~\citet{Garrido2009}, which highlights a potential gap in monitoring learning routes. Secondly, the development of effective ITS is complex and resource-intensive, demanding substantial expertise in algorithm development and content creation, as noted by~\citet{Corbett1997}. The intricate nature of ITS development contributes to associated costs, posing challenges to widespread adoption. Additionally, akin to adaptive learning, the efficacy of ITS is contingent on technology access, which may not be uniformly available to all students, introducing a potential barrier to equitable utilisation across diverse student populations, as emphasised by~\citet{Kabudi2021}. These considerations underscore the need for a nuanced approach in addressing both the potentials and challenges associated with Intelligent Tutoring Systems in educational contexts.

\subsubsection{Theoretical Framework} 
In automated planning, the use of predicates and objects allow us to represent and reason about the world, what tasks may need to be carried out in that world, and what goals are achievable in the context of that world.
A predicate $\predicate\in\predicates$ is denoted by an n-ary predicate symbol, applied to a sequence of zero or more terms $\term_0,\dots,\term_n$. Terms are either constants or variables. We can refer to ground predicates, which represent logical values according to some interpretation as facts, which are divided into two types: positive and negated; as well as having constants for truth and false ($\top$ and $\bot$ respectively).
Now that we have defined a formal element for individual facts of the environment, we aggregate those which are relevant into a state.
A state $\states$ is a finite set of positive facts $\fact$ which follow the closed-world assumption. It follows that $\fact$ is true in $\states$ if $\fact\in\states$. We are also able to assume a simple inference relation; such that $\states\models\fact$ \textit{iff} $\fact\in\states,\states\not\models\fact$ \textit{iff} $\fact\not\in\states$, and $\states\models\bigwedge^{\size{\facts}}_{n=0}\fact_n$ \textit{iff} $\set{\fact_0,\dots,\fact_n}\supseteq\states$.

Planning domains describe the dynamics of an environment by employing operators characterised by a restricted first-order logical representation, creating schemata for actions aimed at altering the state. A schema is a template that offers a broad overview of task completion. It acts as a blueprint for a series of actions in different scenarios, allowing for the capture of common strategies for achieving goals. These schema templates promote knowledge reuse and adaptation, making problem-solving more efficient by tailoring abstract plans to the current context.
An operator $\action$ is represented through a tuple $\tuple{\name(\action),\pre(\action),\eff(\action)}$; where $\name(\action)$ represents the signature or description of $\action$; $\pre(\action)$ describes the preconditions of $\action$~and $\eff(\action)$ represents the effects of~$\action$. $\eff(\action)^+$ and $\eff(\action)^-$ divides $\eff(\action)$ into an add-list of positive predicates and a delete-list of negated predicates respectively. An action is a ground operator instantiated over its free variables. 
An action $\action$ is applicable to \states~if and only if $\states\models\pre(\action)$, and generates a new state $\states'$, such that $\states'\coloneqq(\states\cup\eff(\action)^+/\eff(\action)^-)$. 
A pair $\tuple{\facts,\actions}$ represents a planning domain definition $\planningdomain$, specifying the knowledge of the domain model, consisting of a finite set of facts and a finite set of actions (represented by $\facts$ and $\actions$ respectively).
Planning instances comprise both the planning domain and the planning problem, describing a finite set of objects of the environment. These are the initial state, and the goal state, which represent where the agent(s) would start and finish.

A planning instance $\planninginstance$ is represented by a tuple $\tuple{\planningdomain,\initial,\goals}$, whereby $\planningdomain=\tuple{\facts,\actions}$ represents the domain definition; $\initial\supseteq\facts$ is the initial state specification, defined through specifying the value for all facts in the initial state; and $\goals\supseteq\facts$ is the goal state specification, representing a singular goal, or a set of goals (that may be extremely large) satisfying the goal specification of the desired state to be achieved.
Classical planning representations often separate the definition of the initial state ($\initial$) and the goal state ($\goals$) as part of a planning problem ($\planningdomain$) to be utilised with a domain, such as STRIPS~\cite{Fikes1971} or PDDL~\cite{PDDL1998}.

For a planning instance $\planninginstance=\tuple{\planningdomain,\initial,\goals}$, a plan $\plan$ is a sequence of actions $\tuple{\action_0,\dots,\action_n}$ modifying the initial state $\initial$ into a state $\states\models\goals$, whereby the goal state $\goals$ holds by the successive execution of actions in a plan $\plan$. A plan is optimal ($\optimalplan$) with a length $\vert\optimalplan\vert$ if there exists no other plan $\plan'$ for $\planninginstance~\vert~ \plan'<\plan$.
Actions may have an associated cost applied to them. If an action does not have an associated cost, it is assumed to  have the unit cost ($1$) assigned to it. 
A plan ($\plan$) is optimal if its cost and respective length meets the evaluation criterion \textemdash~whether that's minimal, maximal, or otherwise.

A goal recognition task is a tuple $\recognitionproblem=\tuple{\planningdomain,\initial,\goals,\observationsequence}$; such that $\planningdomain=\tuple{\facts,\actions}$ is a planning domain definition; $\initial$ is the initial state; $\goals$ is the set of possible goals, also assuming the inclusion of some correct hidden goal $\hiddengoal$ ($\hiddengoal\in\goals$); and $\observationsequence=\tuple{\observation_0,\dots,\observation_n}$ is an observation sequence of executed actions. The observation sequence $\observationsequence$ is satisfied by a valid plan $\plan$, following that with each observation $\observation_i=\name(\action), \action\in\actions$, with $\action$ being part of $\plan$, and that plan $\plan$ transitioning $\initial$ into $\hiddengoal$ through the sequential execution of actions in $\plan$.
The optimal solution to a goal recognition task may consist of a singleton set that exclusively contains the correct hidden goal ($\hiddengoal\in\goals$) achieved by the observation sequence $\observationsequence$ of a plan execution.  In this context, we consider a solution suboptimal if it includes more than one goal, as all such goals are equally probable.

The solution to a goal recognition task $\recognitionproblem=\tuple{\planningdomain,\initial,\goals,\observationsequence}$ is a non-empty subset of the set of possible goals $\goalsolution\supseteq G~\vert~\forall G\in\goalsolution$, there exists a plan $\goalplan$ generated from a planning instance, and the observation sequence $\observationsequence$ is satisfied by $\goalplan$.