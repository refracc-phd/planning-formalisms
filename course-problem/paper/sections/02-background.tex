\section{Background}
\begin{enumerate}
    \item \textbf{Overview of related work in the field of using AI and planning for educational assistance}~\citet{Castillo2009} summary,~\cite{Ouyang2023}.
    \item \textbf{The specific challenges faced by secondary education that necessitate innovative solutions}, diversity of learning styles and technological integration of learning solutions \textemdash~like how \citet{Szymkowiak2021} found the inadequacy of conventional education methods for the Z generation, which shows a preference for utilising modern technology in students' learning processes. Facilitating adaptable learning through platforms such as games, mobile apps, videos, and podcasts is pivotal. Their study stresses the importance of supplying these resources and integrating technology into the classroom, as it markedly improves the educational experience for Gen Z students. Acknowledging their shorter attention spans, impatience, and reliance on digital media, educators could adjust to these preferences, as indicated by the questionnaire administered on a prominent peer-to-peer learning platform during the active pursuit of educational insights.
\end{enumerate}

\subsubsection{Adaptive Learning and Intelligent Tutoring Systems} Adaptive learning systems provide personalised learning experiences. They adapt educational content and activities to the specific needs and abilities of each individual learner using AI algorithms. Key features of adaptive learning include:

\begin{itemize}
    \item Personalisation: Adaptive learning systems create a unique learning path for each student by adjusting content and pacing based on real-time performance data, as discussed in~\citet{ElSabagh2021}.
    
    \item Immediate Feedback: These systems offer instant feedback, helping students identify and address mistakes as they occur, leading to a deeper understanding of the material. For instance, in~\citet{munoz2022systematic}, the call for increased rigour and diversity in adaptive learning research is evident from past evaluations. Although there are fewer studies on adaptive learning, their focus is adapting to contextual variations. Most studies have quantitatively assessed adaptive learning interventions, but only a few have delved into understanding the learning progression for adaptation, potentially leaning towards qualitative research. Experimental designs in adaptive learning interventions serve the purpose of identifying causative behaviors. However, future meta-analytic research is crucial to consolidate these findings. Future studies should also investigate other aspects of adaptability, such as the reporting of adaptive techniques or technology used. Notably, despite being excluded from the research sample, only a few meta-analyses were omitted from the study results.
    
    \item Data-Driven Insights: Adaptive learning systems collect and analyse data on student interactions, providing educators with valuable insights into student progress and areas requiring attention.~\citet{Apoki2022} comprehensively evaluates empirical and conceptual research on pedagogical agents in personalised adaptive learning systems, focusing on their adaptive and intelligent roles. While adaptivity and intelligence have historical roots in adaptive hypermedia and intelligent tutoring systems, contemporary trends emphasize the integration of both for resilient systems. The study highlights the strategic use of software agent attributes to overcome the challenges posed by the lack of physical connection in online settings. The findings provide a foundation for future research, offering insights into the potential impact of incorporating pedagogical agents in personalised adaptive learning systems, including anticipated outcomes such as improved performance, task completion, increased motivation, and heightened engagement.
\end{itemize}

\textbf{Limitations:}
\begin{itemize}
    \item Overemphasis on Data: Overreliance on data and algorithms can result in a mechanistic approach to education, potentially overlooking the importance of the human element.~\citet{Rojas2022} covers collaborative problem-solving and providing feedback in real-time, but does not necessarily provide an application to a classroom.
    
    \item Resource Requirements:~\citet{Mirata2020} identifies challenges impeding the widespread adoption of adaptive learning in universities, despite its theoretical advantages. To enhance adoption, further research is urged to address questions about institutional support, necessary skills, and justifying investments. Diverse case studies are recommended to explore general tendencies under varying conditions. The challenges outlined in this study provide a comprehensive framework for organizing and summarizing adaptive learning research. Despite challenges, the potential benefits of adaptive learning, including meeting diverse student needs and improving education quality, warrant extensive research for increased adoption, promising substantial returns for students, institutions, and the regional economy.
\end{itemize}

Intelligent Tutoring Systems simulate one-on-one tutoring experiences for students, using AI planning techniques to provide real-time guidance and adapt instructional approaches. Key features of ITS include:

\begin{itemize}
    \item Real-time Guidance: ITS offers students immediate feedback, clarifications, and guidance as they work through problems or assignments \textemdash~see~\citet{Corbett1997}. In subjects requiring sequential problem-solving, ITS provides step-by-step assistance, helping students grasp complex concepts~\cite{burns2013intelligent}. ITS continuously assesses a student's knowledge and skills, offering targeted practice exercises and content based on their performance.
\end{itemize}

\textbf{Limitations of Intelligent Tutoring Systems}
\begin{itemize}
    \item Lack of Contextual Understanding: ITS may lack a deep understanding of the broader context of learning, potentially leading to recommendations that are technically correct but not pedagogically appropriate.~\citet{Garrido2009} cannot monitor learning routes.
    
    \item Complex Development: Developing effective ITS can be complex and costly, requiring significant resources and expertise in algorithm development and content creation~\cite{Corbett1997}.
    
    \item Technology Access: Like adaptive learning, the effectiveness of ITS relies on technology access, which may not be equally available to all students~\cite{Kabudi2021}.
\end{itemize}

\subsubsection{Theoretical Framework} 
In automated planning, the use of predicates and objects allow us to represent and reason about the world, what tasks may need to be carried out in that world, and what goals are achievable in the context of that world.
A predicate $\predicate\in\predicates$ is denoted by an n-ary predicate symbol, applied to a sequence of zero or more terms $\term_0,\dots,\term_n$. Terms are either constants or variables. We can refer to ground predicates, which represent logical values according to some interpretation as facts, which are divided into two types: positive and negated; as well as having constants for truth and false ($\top$ and $\bot$ respectively).
Now that we have defined a formal element for individual facts of the environment, we aggregate those which are relevant into a state.
A state $\states$ is a finite set of positive facts $\fact$ which follow the closed-world assumption. It follows that $\fact$ is true in $\states$ if $\fact\in\states$. We are also able to assume a simple inference relation; such that $\states\models\fact$ \textit{iff} $\fact\in\states,\states\not\models\fact$ \textit{iff} $\fact\not\in\states$, and $\states\models\bigwedge^{\size{\facts}}_{n=0}\fact_n$ \textit{iff} $\set{\fact_0,\dots,\fact_n}\supseteq\states$.
Planning domains describe the dynamics of an environment by employing operators characterised by a restricted first-order logical representation, creating schemata for actions aimed at altering the state. A schema is a template that offers a broad overview of task completion. It acts as a blueprint for a series of actions in different scenarios, allowing for the capture of common strategies for achieving goals. These schema templates promote knowledge reuse and adaptation, making problem-solving more efficient by tailoring abstract plans to the current context.
An operator $\action$ is represented through a tuple $\tuple{\name(\action),\pre(\action),\eff(\action)}$; where $\name(\action)$ represents the signature or description of $\action$; $\pre(\action)$ describes the preconditions of $\action$~and $\eff(\action)$ represents the effects of~$\action$. $\eff(\action)^+$ and $\eff(\action)^-$ divides $\eff(\action)$ into an add-list of positive predicates and a delete-list of negated predicates respectively. An action is a ground operator instantiated over its free variables. 
An action $\action$ is applicable to \states~if and only if $\states\models\pre(\action)$, and generates a new state $\states'$, such that $\states'\coloneqq(\states\cup\eff(\action)^+/\eff(\action)^-)$. 
A pair $\tuple{\facts,\actions}$ represents a planning domain definition $\planningdomain$, specifying the knowledge of the domain model, consisting of a finite set of facts and a finite set of actions (represented by $\facts$ and $\actions$ respectively).
Planning instances comprise both the planning domain and the planning problem, describing a finite set of objects of the environment. These are the initial state, and the goal state, which represent where the agent(s) would start and finish.
A planning instance $\planninginstance$ is represented by a tuple $\tuple{\planningdomain,\initial,\goals}$, whereby $\planningdomain=\tuple{\facts,\actions}$ represents the domain definition; $\initial\supseteq\facts$ is the initial state specification, defined through specifying the value for all facts in the initial state; and $\goals\supseteq\facts$ is the goal state specification, representing a singular goal, or a set of goals (that may be extremely large) satisfying the goal specification of the desired state to be achieved.
Classical planning representations often separate the definition of the initial state ($\initial$) and the goal state ($\goals$) as part of a planning problem ($\planningdomain$) to be utilised with a domain, such as STRIPS~\cite{Fikes1971} or PDDL~\cite{PDDL1998}.
For a planning instance $\planninginstance=\tuple{\planningdomain,\initial,\goals}$, a plan $\plan$ is a sequence of actions $\tuple{\action_0,\dots,\action_n}$ modifying the initial state $\initial$ into a state $\states\models\goals$, whereby the goal state $\goals$ holds by the successive execution of actions in a plan $\plan$. A plan is optimal ($\optimalplan$) with a length $\vert\optimalplan\vert$ if there exists no other plan $\plan'$ for $\planninginstance~\vert~ \plan'<\plan$.
Actions may have an associated cost applied to them. If an action does not have an associated cost, it is assumed to  have the unit cost ($1$) assigned to it. 
A plan ($\plan$) is optimal if its cost and respective length meets the evaluation criterion \textemdash~whether that's minimal, maximal, or otherwise.
A goal recognition task is a tuple $\recognitionproblem=\tuple{\planningdomain,\initial,\goals,\observationsequence}$; such that $\planningdomain=\tuple{\facts,\actions}$ is a planning domain definition; $\initial$ is the initial state; $\goals$ is the set of possible goals, also assuming the inclusion of some correct hidden goal $\hiddengoal$ ($\hiddengoal\in\goals$); and $\observationsequence=\tuple{\observation_0,\dots,\observation_n}$ is an observation sequence of executed actions. The observation sequence $\observationsequence$ is satisfied by a valid plan $\plan$, following that with each observation $\observation_i=\name(\action), \action\in\actions$, with $\action$ being part of $\plan$, and that plan $\plan$ transitioning $\initial$ into $\hiddengoal$ through the sequential execution of actions in $\plan$.
The optimal solution to a goal recognition task may consist of a singleton set that exclusively contains the correct hidden goal ($\hiddengoal\in\goals$) achieved by the observation sequence $\observationsequence$ of a plan execution.  In this context, we consider a solution suboptimal if it includes more than one goal, as all such goals are equally probable.
The solution to a goal recognition task $\recognitionproblem=\tuple{\planningdomain,\initial,\goals,\observationsequence}$ is a non-empty subset of the set of possible goals $\goalsolution\supseteq G~\vert~\forall G\in\goalsolution$, there exists a plan $\goalplan$ generated from a planning instance, and the observation sequence $\observationsequence$ is satisfied by $\goalplan$.