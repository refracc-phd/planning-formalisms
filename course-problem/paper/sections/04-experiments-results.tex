\section{Experimentation and Results}
\begin{enumerate}
    \item \textbf{Detailed account of the experiments conducted to evaluate the LEARNS system.}
    \item \textbf{Presentation of data and results from the automated planning experiments.} See Table~\ref{tab:experimental-results}.
    \item \textbf{Analysis of the performance of LEARNS in real-world educational scenarios.}
\end{enumerate}

\subsection{Table of Results}
\begin{table*}[ht]
    \centering
    \caption{Experimental Results}\label{tab:experimental-results}
    \begin{tabular}{p{3cm} *{10}{p{1cm}}}
        \toprule
        \multicolumn{1}{c}{\textbf{Heuristic}} & \multicolumn{10}{c}{\textbf{Problems}} \\
        \cmidrule(lr){2-11}
        & \textbf{File 1} & \textbf{File 2} & \textbf{File 3} & \textbf{File 4} & \textbf{File 5} & \textbf{File 6} & \textbf{File 7} & \textbf{File 8} & \textbf{File 9} & \textbf{File 10} \\
        \midrule
        lmcut & \dots & \dots & \dots & \dots & \dots & \dots & \dots & \dots & \dots & \dots \\
        ipdb & \dots & \dots & \dots & \dots & \dots & \dots & \dots & \dots & \dots & \dots \\
        add & \dots & \dots & \dots & \dots & \dots & \dots & \dots & \dots & \dots & \dots \\
        cea & \dots & \dots & \dots & \dots & \dots & \dots & \dots & \dots & \dots & \dots \\
        cegar & \dots & \dots & \dots & \dots & \dots & \dots & \dots & \dots & \dots & \dots \\
        cg & \dots & \dots & \dots & \dots & \dots & \dots & \dots & \dots & \dots & \dots \\
        fastforward & \dots & \dots & \dots & \dots & \dots & \dots & \dots & \dots & \dots & \dots \\
        goalcount & \dots & \dots & \dots & \dots & \dots & \dots & \dots & \dots & \dots & \dots \\
        hm & \dots & \dots & \dots & \dots & \dots & \dots & \dots & \dots & \dots & \dots \\
        hmax & \dots & \dots & \dots & \dots & \dots & \dots & \dots & \dots & \dots & \dots \\
        landmark-cp & \dots & \dots & \dots & \dots & \dots & \dots & \dots & \dots & \dots & \dots \\
        landmark-sum & \dots & \dots & \dots & \dots & \dots & \dots & \dots & \dots & \dots & \dots \\
        merge-shrink & \dots & \dots & \dots & \dots & \dots & \dots & \dots & \dots & \dots & \dots \\
        operator-counting & \dots & \dots & \dots & \dots & \dots & \dots & \dots & \dots & \dots & \dots \\
        fF & \dots & \dots & \dots & \dots & \dots & \dots & \dots & \dots & \dots & \dots \\
        yY & \dots & \dots & \dots & \dots & \dots & \dots & \dots & \dots & \dots & \dots \\
        fFyY & \dots & \dots & \dots & \dots & \dots & \dots & \dots & \dots & \dots & \dots \\
        \bottomrule
    \end{tabular}
\end{table*}



\subsubsection{Goal Recognition}
Classical planning, as discussed in \citet{Fox2003}, treats time as relative and considers only causal dependencies among actions. However, real-world problems often entail complexities, including temporal aspects, numerical values, stochastic effects, and dynamic environments. Numeric planning extends classical planning by incorporating numeric state variables and utilising languages such as PDDL 2.1~\cite{Fox2003} and PDDL+~\cite{Fox2006}. These formalisms enable the representation of time-dependent changes, either as discrete time-dependent effects of durative actions or continuous process-dependent alterations. PDDL+ serves as an extension of PDDL designed specifically to model hybrid systems by integrating continuous processes and events, as outlined by~\citet{Haslum2019}. Its primary purpose is to facilitate the representation of planning domains that combine discrete and continuous elements.

Introduced in \citet{Scala2016} and \citet{ScalaHaslum2016}, the Expressive Numeric Heuristic Search Planner (ENHSP) is compatible with both PDDL 2.1 and PDDL+. ENHSP operates as a forward heuristic search planner, converting PDDL into an asymptotic relaxed planning graph. In this graph, nodes correspond to states explored by the planner, and a heuristic function guides the search process. This function directs the exploration towards nodes whose associated states are reachable from the initial state and bring the system closer to the desired goals.

