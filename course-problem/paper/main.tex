\documentclass[letterpaper]{article}
\usepackage{aaai24}
\usepackage{times}
\usepackage{helvet}
\usepackage{courier}
\usepackage[hyphens]{url}
\usepackage{graphicx}
\usepackage{natbib}
\usepackage{caption}
\usepackage{algorithm}
\usepackage{algorithmic}
\usepackage{newfloat}
\usepackage{listings}
\usepackage{macros}
\usepackage{mathtools}
\usepackage{booktabs}
\usepackage{listings}
\usepackage{parskip}

\urlstyle{rm}
\def\UrlFont{\rm}
\frenchspacing
\setlength{\pdfpagewidth}{8.5in}
\setlength{\pdfpageheight}{11in}

\DeclareCaptionStyle{ruled}{labelfont=normalfont,labelsep=colon,strut=off}

\usepackage{color}

\lstset{ %
  backgroundcolor=\color{white},   % choose the background color; you must add \usepackage{color} or \usepackage{xcolor}
  basicstyle=\footnotesize\ttfamily,        % the size of the fonts that are used for the code
  breakatwhitespace=false,         % sets if automatic breaks should only happen at whitespace
  breaklines=true,                 % sets automatic line breaking
  extendedchars=true,              % lets you use non-ASCII characters; for 8-bits encodings only, does not work with UTF-8                 % adds a frame around the code
  keepspaces=false,                 % keeps spaces in text, useful for keeping indentation of code (possibly needs columns=flexible)
  keywordstyle=\color{blue},       % keyword style
  language=lisp,                 % the language of the code
  otherkeywords={:precondition,:effect,:parameters,:action},            % if you want to add more keywords to the set
  numbers=none,                    % where to put the line-numbers; possible values are (none, left, right)
  showstringspaces=false,          % underline spaces within strings only
  showtabs=false,                  % show tabs within strings adding particular underscores
  tabsize=2,                       % sets default tabsize to 2 spaces
}

\floatstyle{ruled}
\newfloat{listing}{tb}{lst}
\floatname{listing}{Listing}

\pdfinfo{
    /TemplateVersion (2024.1)
}
\setcounter{secnumdepth}{0}

\title{LEARNS:~Learning Enhancement and Recognition Navigation System}

\author{
    Stewart Anderson\textsuperscript{\rm 1},
    Felipe Meneguzzi\textsuperscript{\rm 1}
}

\affiliations{
    \textsuperscript{\rm 1}King's College, University of Aberdeen\\
    School of Natural and Computer Sciences \\
    Department of Computer Science \\
    \{first.last\}@abdn.ac.uk
}
\begin{document}
    \maketitle

    \begin{abstract}
    \end{abstract}

    \section{Introduction}
\begin{enumerate}
    \item \textbf{Introduction to the LEARNS project and its significance.}
    \item \textbf{The current challenges in secondary education and the need for automated planning solutions.}
    \item \textbf{Overview of the goals and objectives.}
    \item \textbf{Brief outline of the structure of the paper and what readers can expect.}
    \item \textbf{Main contributions of the paper.}
\end{enumerate}
    \section{Background}
\begin{enumerate}
    \item \textbf{Background of Planning}~\cite{Scala2016,ScalaHaslum2016}.
    \item \textbf{Existing educational support systems and their limitations}, like how~\citet{Garrido2009} cannot monitor the learning routes, similar to intelligent tutoring systems;~\citet{Rojas2022} covers collaborative problem solving and providing feedback in real-time, but does not necessarily provide an application to a classroom.
    \item \textbf{Theoretical framework and concepts related to automated planning}.
    \item \textbf{Overview of related work in the field of using AI and planning for educational assistance}~\cite{Castillo2009}.
    \item \textbf{The specific challenges faced by secondary education that necessitate innovative solutions.}
\end{enumerate}

    \section{Modelling Secondary Education in PDDL}
This project employs a structured ontology in PDDL, encompassing types, constants, and predicates to model secondary education. Types include courses, course levels, weeks, units, extracurricular activities, grades, preferred learning styles, and student support requirements. Constants consist of strategies, school calendar weeks, units, extracurricular activities, predicted grades, and student preferences and support requirements. Predicates establish relationships such as finished courses, course enrollment, predicted grades, completion of weeks, units, and extracurricular activities, participation in study groups, adoption of learning strategies, identification of support needs, and provision of support.

\subsection{LEARNS Architecture}
Planning learning paths for students requires the initial step of creating a framework, which is one that contains 36 weeks. There are $32$ weeks for learning, and $4$ weeks for students to be assessed by means of unit tests. There are also a number of variables to consider, such as a student (\texttt{?s}), the course they are sitting (\texttt{?c}), the level of the course (\texttt{?l}), what grade they are expected to achieve in a given course (\texttt{?g}), what week they are currently on (\texttt{?w}), what unit they are achieving on (\texttt{?u}), what extracurricular activity they are doing (\texttt{?e}), what learning strategy they prefer (\texttt{?t}), and what support requirements they may have (\texttt{?r}). These variables are used to aid in the guidance of students through this learning recognition system.

The successful completion of each week is systematically recorded through the \texttt{done-week} predicate, as denoted in Figures~\ref{fig:do-week-one} and~\ref{fig:do-week-n}. Figure~\ref{fig:do-week-n} is an action that is chained from Figure~\ref{fig:do-week-one}, and is executed for each week that is to be completed (week-n), $n$ is in the range of $2 \leq n \leq 8$ and represents the weeks $2$ through $8$ respectively. This is formalised in Figures~\ref{fig:do-week-one} and~\ref{fig:do-week-n}.

\begin{figure}[t]
    \begin{lstlisting}
    (:action do-week-one
      :parameters (?s - student ?c - course ?l - course-level)
      :precondition (and 
        (takes-course ?s ?c ?l)
        (not(done-week week-one ?s ?c ?l))
      )
      :effect (and 
        (done-week week-one ?s ?c ?l)
      )
    )
    \end{lstlisting}
    \caption{PDDL action that commences study for the student}\label{fig:pddl-action-do-week-one}
\end{figure}
\begin{figure}[t]
    \begin{lstlisting}
    (:action do-week-n
      :parameters (?s - student ?c - course ?l - course-level)
      :precondition (and 
        (not(done-week week-n ?s ?c ?l))
        (done-week week-(n-1) ?s ?c ?l)
      )
      :effect (and 
        (done-week week-n ?s ?c ?l)
      )
    )
    \end{lstlisting}
    \caption{PDDL action that continues study for the student by permitting them to take the $n$th week of study.}\label{fig:do-week-n}
\end{figure}
    

After eight weeks have passed, students are provided with a unit test to complete. The \texttt{done-unit} predicate in Figures~\ref{fig:take-unit-one} and~\ref{fig:take-unit-n} denote their passing of the unit \textemdash~assuming that all students who take a unit will pass it. This is a simplification of the problem, as it is possible for students to fail a unit test. However, this is not a concern for this project, as the focus is on the planning of learning paths, rather than the planning of resits. Following the accomplishment of a unit, students are then tasked with a subsequent 8-week cycle, this cycle persists until the student has completed all four units from the course.

\begin{figure}[t]
    \begin{lstlisting}
    (:action take-unit-one
      :parameters (?s - student ?c - course ?l - course-level)
      :precondition (and 
        (takes-course ?s ?c ?l)
        (not(done-unit unit-one ?s ?c ?l))
        (done-week week-eight ?s ?c ?l)
      )
      :effect (and
        (done-unit unit-one ?s ?c ?l)
        (not(done-week week-eight ?s ?c ?l))
        (not(done-week week-seven ?s ?c ?l))
        (not(done-week week-six ?s ?c ?l))
        (not(done-week week-five ?s ?c ?l))
        (not(done-week week-four ?s ?c ?l))
        (not(done-week week-three ?s ?c ?l))
        (not(done-week week-two ?s ?c ?l))
        (not(done-week week-one ?s ?c ?l))
      )
    )
    \end{lstlisting}
    \caption{PDDL action that allows a student to take unit one}\label{fig:pddl-action-take-unit-one}
\end{figure}
\begin{figure}[t]
  \small  % Adjust font size
  \begin{align*}
  &\pre(\text{take-unit-n}) = \\
  &\quad \text{takes-course}(\text{?s}, \text{?c}, \text{?l}) \land \\
  &\quad \lnot \text{done-unit}(\text{unit-n}, \text{?s}, \text{?c}, \text{?l}) \land \\
  &\quad \text{done-unit}(\text{unit-(n-1)}, \text{?s}, \text{?c}, \text{?l}) \land \\
  &\quad \text{done-week}(\text{week-eight}, \text{?s}, \text{?c}, \text{?l}) \\
  &\eff(\text{take-unit-n}) = \\
  &\quad \text{done-unit}(\text{unit-n}, \text{?s}, \text{?c}, \text{?l}) \land \\
  &\quad \lnot \text{done-week}(\text{week-eight}, \text{?s}, \text{?c}, \text{?l}) \land \\
  &\quad \lnot \text{done-week}(\text{week-seven}, \text{?s}, \text{?c}, \text{?l}) \land \\
  &\quad \lnot \text{done-week}(\text{week-six}, \text{?s}, \text{?c}, \text{?l}) \land \\
  &\quad \lnot \text{done-week}(\text{week-five}, \text{?s}, \text{?c}, \text{?l}) \land \\
  &\quad \lnot \text{done-week}(\text{week-four}, \text{?s}, \text{?c}, \text{?l}) \land \\
  &\quad \lnot \text{done-week}(\text{week-three}, \text{?s}, \text{?c}, \text{?l}) \land \\
  &\quad \lnot \text{done-week}(\text{week-two}, \text{?s}, \text{?c}, \text{?l}) \land \\
  &\quad \lnot \text{done-week}(\text{week-one}, \text{?s}, \text{?c}, \text{?l})
  \end{align*}
  \caption{Logical formalisation of the PDDL action for a student taking the $n$th unit}\label{fig:take-unit-n}
\end{figure}

A student has completed a course when all four units are complete, and is denoted by the \texttt{finished-course} predicate. Figure~\ref{fig:finish-course} provides the action to check that the student has completed the course. This is a prerequisite for the student to be able to take the final exam \textemdash~of which is not monitored by this system, as it is an application of automated planning to generate learning paths for students with additional support requirements, not for handling the logistics of the school.

\begin{figure}[t]
  \small  % Adjust font size
  \begin{align*}
  &\pre(\text{finish-course}) = \\
  &\quad \text{done-unit}(\text{unit-four}, \text{?s}, \text{?c}, \text{?l}) \land \\
  &\quad \text{takes-course}(\text{?s}, \text{?c}, \text{?l}) \\
  &\eff(\text{finish-course}) = \\
  &\quad \text{finished-course}(\text{?s}, \text{?c}, \text{?l})
  \end{align*}
  \caption{Logical formalisation of the PDDL action for finishing a course}\label{fig:finish-course}
\end{figure}

Alongside their academics, extracurricular activities play a vital role in student development. For each unit, students can participate in two extracurricular activities. Grade-a students are expected to commit to both, while grade-b students typically commit to one. Figures~\ref{fig:extra-curricular-one} and~\ref{fig:extra-curricular-n} illustrate this academic involvement, highlighting the sequential nature of these activities.

\begin{figure}[t]
    \begin{lstlisting}
    (:action do-extra-curricular-one
      :parameters (?s - student ?c - course ?l - course-level)
      :precondition (and 
        (takes-course ?s ?c ?l)
        (or
          (grade a ?s ?c ?l)
          (grade b ?s ?c ?l)
        )
        (not
          (done-extra-curricular ec-one ?s ?c ?l)
        )
      )
      :effect (and 
        (done-extra-curricular ec-one ?s ?c ?l)
      )
    )
    \end{lstlisting}
    \caption{PDDL action to allow a student to take the first extracurricular activity}\label{fig:extra-curricular-one}
\end{figure}
\begin{figure}[t]
    \begin{lstlisting}
    (:action do-extra-curricular-n
      :parameters (?s - student ?c - course ?l - course-level)
      :precondition (and 
        (takes-course ?s ?c ?l)
        (grade a ?s ?c ?l)
        (not
          (done-extra-curricular ec-n ?s ?c ?l)
        )
        (done-extra-curricular ec-(n-1) ?s ?c ?l)
      )
      :effect (and 
        (done-extra-curricular ec-n ?s ?c ?l)
      )
    )
    \end{lstlisting}
    \caption{PDDL action that has a student to the $n$th extracurricular activity}\label{fig:extra-curricular-n}
\end{figure}

Extracurricular activities alternate, starting with \texttt{do-extra-curricular-one} and progressing to \texttt{do-extra-curricular-n}. This pattern repeats for each unit, and completion is indicated by the \texttt{done-extra-curricular} predicate. The approximation stems from the earlier statement that grade-a or grade-b students may engage in either one or two extracurricular activities. However, this simplification overlooks the possibility of non-grade-a or non-grade-b students participating and grade-a or grade-b students deviating from the specified number of extracurricular engagements. However, this system is designed to provide recommendations to aid in the planning of learning paths. 

Students willing to engage in collaborative group dynamics can participate in group-oriented extracurricular activities. This collaboration is depicted in Figures~\ref{fig:team-extra-curricular-one} and~\ref{fig:team-extra-curricular-n}, illustrating the synergistic environment fostered through teamwork in extracurricular pursuits.

\begin{figure}[t]
  \small  % Adjust font size
  \begin{align*}
  &\pre(\text{do-team-extra-curricular-one}) = \\
  &\quad \text{takes-course}(\text{?s}, \text{?c}, \text{?l}) \land \\
  &\quad (\text{grade a}(\text{?s}, \text{?c}, \text{?l}) \lor \\
  &\quad \text{grade b}(\text{?s}, \text{?c}, \text{?l})) \land \\
  &\quad \text{study-group}(\text{?c}, \text{?l}) \land \\
  &\quad \text{uses-strategy}(\text{?s}, \text{teamwork}) \land \\
  &\quad \lnot \text{done-extra-curricular}(\text{ec-one}, \text{?s}, \text{?c}, \text{?l}) \\
  &\eff(\text{do-team-extra-curricular-one}) = \\
  &\quad \text{done-extra-curricular}(\text{ec-one}, \text{?s}, \text{?c}, \text{?l})
  \end{align*}
  \caption{Logical formalisation of the PDDL action for a student doing the first extracurricular activity as part of a team}\label{fig:team-extra-curricular-one}
\end{figure}
\begin{figure}[t]
    \begin{lstlisting}
    (:action do-team-extra-curricular-n
      :parameters (?s - student ?c - course ?l - course-level)
      :precondition (and 
        (takes-course ?s ?c ?l)
        (grade a ?s ?c ?l)
        (not
          (done-extra-curricular ec-n ?s ?c ?l)
        )
        (study-group ?c ?l)
        (uses-strategy ?s teamwork)
        (done-extra-curricular ec-(n-1) ?s ?c ?l)
      )
      :effect (and 
        (done-extra-curricular ec-n ?s ?c ?l)
      )
    )
    \end{lstlisting}
    \caption{PDDL action to allow a student to do the $n$th extracurricular activity as part of a team.}\label{fig:team-extra-curricular-n}
    \end{figure}

\subsection{Support Structures}
Through our independent research, we have identified that students grappling with communication challenges can benefit significantly from specialised workshops aimed at enhancing their skills. This provision is particularly pertinent for students facing social issues, those on the autism spectrum, or individuals with linguistic difficulties. Moreover, students who prefer specific learning styles, such as teamwork, project-based learning, or blended learning for those unable to attend physical school, can find valuable support in communication improvement workshops. Figure~\ref{fig:improve-comms-workshop} illustrates the relevance and potential impact of these workshops.

\begin{figure}[t]
    \begin{lstlisting}
    (:action recommend-improving-communications-workshop
      :parameters (?s - student)
      :precondition (and
        (or
          (has-support-need ?s asc-asd)
          (has-support-need ?s language)
          (uses-strategy ?s teamwork)
          (uses-strategy ?s project-based)
          (uses-strategy ?s blended-learning)
        )
      )
      :effect (and 
        (given-support ?s improving-comms-workshop)
      )
    )
    \end{lstlisting}
    \caption{PDDL action to recommend a student attends a workshop on improving their communication skills.}\label{fig:improve-comms-workshop}
\end{figure}

For students encountering difficulties in reading or those who could benefit from tools like audiobooks, we offer a dedicated reading group workshop. The details of this initiative are outlined in Figure~\ref{fig:reading-group}, providing a comprehensive understanding of the resources and strategies employed to assist students in enhancing their reading abilities.

\begin{figure}[t]
    \begin{lstlisting}
    (:action recommend-reading-group
      :parameters (?s - student)
      :precondition (and 
        (has-support-need ?s language)
        (has-support-need ?s blind-visual)
      )
      :effect (and 
        (given-support ?s reading-group)
      )
    )
    \end{lstlisting}
    \caption{PDDL action to recommend a student attends a reading group.}\label{fig:pddl-action-reading-group}
\end{figure}

Recognising the diverse needs of our student population, we acknowledge the significance of technology-assisted learning. We are equipped to provide students with tailored tools to support their learning journeys. This could be especially beneficial for students on the autism spectrum, facing language difficulties, experiencing profound deafness or hearing impairments, those with visual impairments, or individuals who simply prefer leveraging technology for their educational pursuits. Figure~\ref{fig:tech-assist} offers an insight into the specifics of our approach in this domain.

\begin{figure}[t]
    \begin{lstlisting}
    (:action recommend-technological-assistance
      :parameters (?s - student)
      :precondition (and 
        (or
          (has-support-need ?s asc-asd)
          (has-support-need ?s language)
          (has-support-need ?s deaf-hearing)
          (has-support-need ?s blind-visual)
          (uses-strategy ?s technological-tools)
          (uses-strategy ?s blended-learning)
        )
      )
      :effect (and 
        (given-support ?s tech-assist)
      )
    )
    \end{lstlisting}
    \caption{PDDL action to recommend a student receives a digitised format to assist with their learning.}\label{fig:tech-assist}
\end{figure}

Furthermore, we understand that some students may derive greater benefits from breaking their study time into smaller, focused intervals. Embracing the Pomodoro technique, students can optimise their learning experiences by strategically dividing their study sessions. Figure~\ref{fig:pomo-study} provides a detailed illustration of how this technique can be applied to enhance learning outcomes.

\begin{figure}[t]
    \begin{lstlisting}
    (:action recommend-pomodoro-study
      :parameters (?s - student)
      :precondition (and 
        (or
          (has-support-need ?s asc-asd)
          (has-support-need ?s tourettes)
          (uses-strategy ?s pomodoro)
          (uses-strategy ?s teamwork)
        )
      )
      :effect (and 
        (given-support ?s pomo)
      )
    )
    \end{lstlisting}
    \caption{PDDL Action: recommend-pomodoro-study}\label{fig:pomo-study}
    \end{figure}
    \section{Experimentation and Results}
\begin{enumerate}
    \item \textbf{Detailed account of the experiments conducted to evaluate the LEARNS system.}
    \item \textbf{Presentation of data and results from the automated planning experiments.} See Table~\ref{tab:experimental-results}.
    \item \textbf{Analysis of the performance of LEARNS in real-world educational scenarios.}
\end{enumerate}

\subsection{Table of Results}
\begin{table*}[ht]
    \centering
    \caption{Experimental Results}\label{tab:experimental-results}
    \begin{tabular}{p{3cm} *{10}{p{1cm}}}
        \toprule
        \multicolumn{1}{c}{\textbf{Heuristic}} & \multicolumn{10}{c}{\textbf{Problems}} \\
        \cmidrule(lr){2-11}
        & \textbf{File 1} & \textbf{File 2} & \textbf{File 3} & \textbf{File 4} & \textbf{File 5} & \textbf{File 6} & \textbf{File 7} & \textbf{File 8} & \textbf{File 9} & \textbf{File 10} \\
        \midrule
        lmcut & \dots & \dots & \dots & \dots & \dots & \dots & \dots & \dots & \dots & \dots \\
        ipdb & \dots & \dots & \dots & \dots & \dots & \dots & \dots & \dots & \dots & \dots \\
        add & \dots & \dots & \dots & \dots & \dots & \dots & \dots & \dots & \dots & \dots \\
        cea & \dots & \dots & \dots & \dots & \dots & \dots & \dots & \dots & \dots & \dots \\
        cegar & \dots & \dots & \dots & \dots & \dots & \dots & \dots & \dots & \dots & \dots \\
        cg & \dots & \dots & \dots & \dots & \dots & \dots & \dots & \dots & \dots & \dots \\
        fastforward & \dots & \dots & \dots & \dots & \dots & \dots & \dots & \dots & \dots & \dots \\
        goalcount & \dots & \dots & \dots & \dots & \dots & \dots & \dots & \dots & \dots & \dots \\
        hm & \dots & \dots & \dots & \dots & \dots & \dots & \dots & \dots & \dots & \dots \\
        hmax & \dots & \dots & \dots & \dots & \dots & \dots & \dots & \dots & \dots & \dots \\
        landmark-cp & \dots & \dots & \dots & \dots & \dots & \dots & \dots & \dots & \dots & \dots \\
        landmark-sum & \dots & \dots & \dots & \dots & \dots & \dots & \dots & \dots & \dots & \dots \\
        merge-shrink & \dots & \dots & \dots & \dots & \dots & \dots & \dots & \dots & \dots & \dots \\
        operator-counting & \dots & \dots & \dots & \dots & \dots & \dots & \dots & \dots & \dots & \dots \\
        fF & \dots & \dots & \dots & \dots & \dots & \dots & \dots & \dots & \dots & \dots \\
        yY & \dots & \dots & \dots & \dots & \dots & \dots & \dots & \dots & \dots & \dots \\
        fFyY & \dots & \dots & \dots & \dots & \dots & \dots & \dots & \dots & \dots & \dots \\
        \bottomrule
    \end{tabular}
\end{table*}



\subsubsection{Goal Recognition}
Classical planning, as discussed in \citet{Fox2003}, treats time as relative and considers only causal dependencies among actions. However, real-world problems often entail complexities, including temporal aspects, numerical values, stochastic effects, and dynamic environments. Numeric planning extends classical planning by incorporating numeric state variables and utilising languages such as PDDL 2.1~\cite{Fox2003} and PDDL+~\cite{Fox2006}. These formalisms enable the representation of time-dependent changes, either as discrete time-dependent effects of durative actions or continuous process-dependent alterations. PDDL+ serves as an extension of PDDL designed specifically to model hybrid systems by integrating continuous processes and events, as outlined by~\citet{Haslum2019}. Its primary purpose is to facilitate the representation of planning domains that combine discrete and continuous elements.

Introduced in \citet{Scala2016} and \citet{ScalaHaslum2016}, the Expressive Numeric Heuristic Search Planner (ENHSP) is compatible with both PDDL 2.1 and PDDL+. ENHSP operates as a forward heuristic search planner, converting PDDL into an asymptotic relaxed planning graph. In this graph, nodes correspond to states explored by the planner, and a heuristic function guides the search process. This function directs the exploration towards nodes whose associated states are reachable from the initial state and bring the system closer to the desired goals.


    \section{Conclusions and Future Work}
\begin{enumerate}
    \item \textbf{Detailed account of the experiments conducted to evaluate the SEAPS system.}
    \item \textbf{Presentation of data and results from the automated planning experiments.}
    \item \textbf{Analysis of the performance of SEAPS in real-world educational scenarios.}
    \item \textbf{Comparison of SEAPS outcomes with traditional educational support methods.}
    \item \textbf{Implications of the results and their significance for the educational landscape in Scotland.}
\end{enumerate}

    
    % What are the main contributions?

    % Do we really want to do goal recognition at this point? This is plan recognition for multiple agents (being a series of single-agent planning problems with the possibility to have them be co-operative if possible) - isn't it?

    % In goal rec, paragraph about ENHSP but don't mention specific planner until the experimentation.

    % What kind of actions can be taken from the timetabling problem? We have a weekly granularity -- we can't do anything about this at this stage. What would we require in the model to meaningfuly do this.
    
    % How can we make this slightly more complex so it is a challenge for planning/goal recognition.
    
    % Showcase the meaningful problem.
    
    % Check planning problems for PDDL -- I've contacted Eva Onaindia about this, to no avail.
    
    % what happens if a student fails? hypothesis for resit? - I get why, but wouldn't that just be common sense? Wouldn't it be advisable to assume that would be the case in the first place?
    
    % unidentified support need??? can we develop this into something that is not strictly another timetabling problems? -- this is a good idea, but we need to be careful about how we do this. An unidentified support need would imply that we know enough about the student to know that they may be at risk of possibly having one, so would it be an opinion that the student itself would have to have?

    % What if we could compare the results of the discrete plan against the continuous one? How would that faire? 

    % Story of helping students -- provide additional insights against other formalisations.

    % List of elements that make the domain more complex -- we can say as to why what we have will have more impact.
    % Agents have non-overlapping sets of actions and local-sensing in a sense. 

    % Put PDDL as supplementary material; replace precons and effects with straight logic - if appropriate.


    \bibliography{aaai24}
\end{document}
