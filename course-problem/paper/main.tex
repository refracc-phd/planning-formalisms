\documentclass[letterpaper]{article}
\usepackage{aaai24}
\usepackage{times}
\usepackage{helvet}
\usepackage{courier}
\usepackage[hyphens]{url}
\usepackage{graphicx}
\usepackage{natbib}
\usepackage{caption}
\usepackage{algorithm}
\usepackage{algorithmic}
\usepackage{newfloat}
\usepackage{listings}
\usepackage{macros}
\usepackage{mathtools}

\urlstyle{rm}
\def\UrlFont{\rm}
\frenchspacing
\setlength{\pdfpagewidth}{8.5in}
\setlength{\pdfpageheight}{11in}

\DeclareCaptionStyle{ruled}{labelfont=normalfont,labelsep=colon,strut=off}

\lstset{
    basicstyle={\footnotesize\ttfamily},
    numbers=left,
    numberstyle=\footnotesize,
    xleftmargin=2em,
    aboveskip=0pt,
    belowskip=0pt,
    showstringspaces=false,
    tabsize=2,
    breaklines=true
}

\floatstyle{ruled}
\newfloat{listing}{tb}{lst}
\floatname{listing}{Listing}

\pdfinfo{
    /TemplateVersion (2024.1)
}
\setcounter{secnumdepth}{0}

\title{LEARNS:~Learning Enhancement and Recognition Navigation System}

\author{
    Stewart Anderson\textsuperscript{\rm 1},
    Felipe Meneguzzi\textsuperscript{\rm 1}
}

\affiliations{
    \textsuperscript{\rm 1}King's College, University of Aberdeen\\
    School of Natural and Computer Sciences \\
    Department of Computer Science \\
    \{first.last\}@abdn.ac.uk
}
\begin{document}
    \maketitle

    \begin{abstract}
    \end{abstract}

    \section{Introduction}
\begin{enumerate}
    \item Introduction to the SEAPS project and its significance in the context of secondary education in Scotland.
    \item The current challenges in secondary education and the need for automated planning solutions.
    \item Overview of the goals and objectives of the SEAPS (Secondary Education Assistance Planning Scheme).
    \item Brief outline of the structure of the paper and what readers can expect.
    \item The role of automated planning in addressing educational challenges and enhancing student support.
\end{enumerate}
    \section{Background}
\begin{enumerate}
    \item Background of Planning (ENHSP).
    \item Existing educational support systems and their limitations~\cite{Castillo2009}.
    \item Theoretical framework and concepts related to automated planning.
    \item Overview of related work in the field of using AI and planning for educational assistance.
    \item The specific challenges faced by Scottish secondary education that necessitate innovative solutions.
\end{enumerate}

    \section{Modelling Secondary Education in PDDL}
This project employs a structured ontology in PDDL, encompassing types, constants, and predicates to model secondary education. Types include courses, course levels, weeks, units, extracurricular activities, grades, preferred learning styles, and student support requirements. Constants consist of strategies, school calendar weeks, units, extracurricular activities, predicted grades, and student preferences and support requirements. Predicates establish relationships such as finished courses, course enrollment, predicted grades, completion of weeks, units, and extracurricular activities, participation in study groups, adoption of learning strategies, identification of support needs, and provision of support.

\subsection{LEARNS Architecture}
Planning learning paths for students requires the initial step of creating a framework, which is one that contains 36 weeks. There are $32$ weeks for learning, and $4$ weeks for students to be assessed by means of unit tests. There are also a number of variables to consider, such as a student (\texttt{?s}), the course they are sitting (\texttt{?c}), the level of the course (\texttt{?l}), what grade they are expected to achieve in a given course (\texttt{?g}), what week they are currently on (\texttt{?w}), what unit they are achieving on (\texttt{?u}), what extracurricular activity they are doing (\texttt{?e}), what learning strategy they prefer (\texttt{?t}), and what support requirements they may have (\texttt{?r}). These variables are used to aid in the guidance of students through this learning recognition system.

The successful completion of each week is systematically recorded through the \texttt{done-week} predicate, as denoted in Figures~\ref{fig:do-week-one} and~\ref{fig:do-week-n}. Figure~\ref{fig:do-week-n} is an action that is chained from Figure~\ref{fig:do-week-one}, and is executed for each week that is to be completed (week-n), $n$ is in the range of $2 \leq n \leq 8$ and represents the weeks $2$ through $8$ respectively. This is formalised in Figures~\ref{fig:do-week-one} and~\ref{fig:do-week-n}.

\begin{figure}[t]
    \begin{align*}
    &\precondition(\text{do-week-one}) = \\
    &\quad ( \text{takes-course}(\text{?s}, \text{?c}, \text{?l}) \\
    &\quad \land \lnot \text{done-week}(\text{week-one}, \text{?s}, \text{?c}, \text{?l}) ) \\
    &\eff(\text{do-week-one}) = \text{done-week}(\text{week-one}, \text{?s}, \text{?c}, \text{?l})
    \end{align*}
    \caption{Logical formalization of the PDDL action for commencing study}\label{fig:formalised-do-week-one}
\end{figure}
\begin{figure}[t]
    \begin{lstlisting}
    (:action do-week-n
      :parameters (?s - student ?c - course ?l - course-level)
      :precondition (and 
        (not(done-week week-n ?s ?c ?l))
        (done-week week-(n-1) ?s ?c ?l)
      )
      :effect (and 
        (done-week week-n ?s ?c ?l)
      )
    )
    \end{lstlisting}
    \caption{PDDL action that continues study for the student by permitting them to take the $n$th week of study.}\label{fig:pddl-action-do-week-n}
\end{figure}
    

After eight weeks have passed, students are provided with a unit test to complete. The \texttt{done-unit} predicate in Figures~\ref{fig:take-unit-one} and~\ref{fig:take-unit-n} denote their passing of the unit \textemdash~assuming that all students who take a unit will pass it. This is a simplification of the problem, as it is possible for students to fail a unit test. However, this is not a concern for this project, as the focus is on the planning of learning paths, rather than the planning of resits. Following the accomplishment of a unit, students are then tasked with a subsequent 8-week cycle, this cycle persists until the student has completed all four units from the course.

\begin{figure}[t]
    \begin{lstlisting}
    (:action take-unit-one
      :parameters (?s - student ?c - course ?l - course-level)
      :precondition (and 
        (takes-course ?s ?c ?l)
        (not(done-unit unit-one ?s ?c ?l))
        (done-week week-eight ?s ?c ?l)
      )
      :effect (and
        (done-unit unit-one ?s ?c ?l)
        (not(done-week week-eight ?s ?c ?l))
        (not(done-week week-seven ?s ?c ?l))
        (not(done-week week-six ?s ?c ?l))
        (not(done-week week-five ?s ?c ?l))
        (not(done-week week-four ?s ?c ?l))
        (not(done-week week-three ?s ?c ?l))
        (not(done-week week-two ?s ?c ?l))
        (not(done-week week-one ?s ?c ?l))
      )
    )
    \end{lstlisting}
    \caption{PDDL action that allows a student to take unit one}\label{fig:pddl-action-take-unit-one}
\end{figure}
\input{figures/tex/actions/unit-n.tex}

A student has completed a course when all four units are complete, and is denoted by the \texttt{finished-course} predicate. Figure~\ref{fig:finish-course} provides the action to check that the student has completed the course. This is a prerequisite for the student to be able to take the final exam \textemdash~of which is not monitored by this system, as it is an application of automated planning to generate learning paths for students with additional support requirements, not for handling the logistics of the school.

Alongside their academics, extracurricular activities play a vital role in student development. For each unit, students can participate in two extracurricular activities. Grade-a students are expected to commit to both, while grade-b students typically commit to one. Figures~\ref{fig:extra-curricular-one} and~\ref{fig:extra-curricular-n} illustrate this academic involvement, highlighting the sequential nature of these activities.

\begin{figure}[t]
  \small  % Adjust font size
  \begin{align*}
  &\pre(\text{do-extra-curricular-one}) = \\
  &\quad \text{takes-course}(\text{?s}, \text{?c}, \text{?l}) \land \\
  &\quad (\text{grade a}(\text{?s}, \text{?c}, \text{?l}) \lor \text{grade b}(\text{?s}, \text{?c}, \text{?l})) \land \\
  &\quad \lnot \text{done-extra-curricular}(\text{ec-one}, \text{?s}, \text{?c}, \text{?l}) \\
  &\eff(\text{do-extra-curricular-one}) = \\
  &\quad \text{done-extra-curricular}(\text{ec-one}, \text{?s}, \text{?c}, \text{?l})
  \end{align*}
  \caption{Logical formalisation of the PDDL action for the first extracurricular activity}\label{fig:extra-curricular-one}
\end{figure}
\begin{figure}[t]
  \small  % Adjust font size
  \begin{align*}
  &\pre(\text{do-extra-curricular-n}) = \\
  &\quad \text{takes-course}(\text{?s}, \text{?c}, \text{?l}) \land \\
  &\quad \text{grade}(\text{a}, \text{?s}, \text{?c}, \text{?l}) \land \\
  &\quad \lnot \text{done-extra-curricular}(\text{ec-n}, \text{?s}, \text{?c}, \text{?l}) \land \\
  &\quad \text{done-extra-curricular}(\text{ec-(n-1)}, \text{?s}, \text{?c}, \text{?l}) \\
  &\eff(\text{do-extra-curricular-n}) = \\
  &\quad \text{done-extra-curricular}(\text{ec-n}, \text{?s}, \text{?c}, \text{?l})
  \end{align*}
  \caption{Logical formalisation of the PDDL action for the $n$th extracurricular activity}\label{fig:extra-curricular-n}
\end{figure}

Extracurricular activities alternate, starting with \texttt{do-extra-curricular-one} and progressing to (roughly) \texttt{do-extra-curricular-n}. This pattern repeats for each unit, and completion is indicated by the \texttt{done-extra-curricular} predicate. The approximation stems from the earlier statement that grade-a or grade-b students may engage in either one or two extracurricular activities. However, this simplification overlooks the possibility of non-grade-a or non-grade-b students participating and grade-a or grade-b students deviating from the specified number of extracurricular engagements.

Students willing to engage in collaborative group dynamics can participate in group-oriented extracurricular activities. This collaboration is depicted in Figures~\ref{fig:team-extra-curricular-one} and~\ref{fig:team-extra-curricular-n}, illustrating the synergistic environment fostered through teamwork in extracurricular pursuits.

\subsection{Support Structures}
\begin{itemize}
    \item We recognise through our independent research that if a student is struggling with their communication skills, it may be possible to provide a workshop to help them improve. This could be for students who have social issues, are on the autism spectrum or may have linguistic issues. Additionally, a student could also prefer to use specific learning styles that require aid of a communications improvement workshop \textemdash~such as teamwork, project-based learning or blended learning for those that are unable to be physically in school. See Figure~\ref{fig:improve-comms-workshop}.
    \item Should a student face particular difficulty with their reading ability, or may benefit from using tools such as audiobooks, we can provide a reading group workshop to help them with that. This is detailed in Figure~\ref{fig:reading-group}.
    \item If a student prefers to learn with the assistance of technology, we can provide them with a tool to help them with their learning, which could be for a variety of reasons. These reasons could include the student being on the autism spectrum, may have difficulties with language, could be profoundly deaf or hard of hearing, could be fully blind or visually impaired, or they may prefer to just use technology to assist their learning or be learning from a distance. This is detailed in Figure~\ref{fig:tech-assist}.
    \item Additionally, students may prefer and/or benefit from breaking up their study time into smaller chunks, which is known as the Pomodoro technique. This is detailed in Figure~\ref{fig:pomo-study}.
\end{itemize}
    \section{Experimentation and Results}
\begin{enumerate}
    \item \textbf{Detailed account of the experiments conducted to evaluate the LEARNS system.}
    \item \textbf{Presentation of data and results from the automated planning experiments.} See Table~\ref{tab:experimental-results}.
    \item \textbf{Analysis of the performance of LEARNS in real-world educational scenarios.}
\end{enumerate}

\subsection{Table of Results}

\begin{table*}[ht]
    \centering
    \caption{Experimental Results}\label{tab:experimental-results}
    \begin{tabular}{p{3cm} *{10}{p{1cm}}}
        \toprule
        \multicolumn{1}{c}{\textbf{Heuristic}} & \multicolumn{10}{c}{\textbf{Problems}} \\
        \cmidrule(lr){2-11}
        & \textbf{File 1} & \textbf{File 2} & \textbf{File 3} & \textbf{File 4} & \textbf{File 5} & \textbf{File 6} & \textbf{File 7} & \textbf{File 8} & \textbf{File 9} & \textbf{File 10} \\
        \midrule
        lmcut & \dots & \dots & \dots & \dots & \dots & \dots & \dots & \dots & \dots & \dots \\
        ipdb & \dots & \dots & \dots & \dots & \dots & \dots & \dots & \dots & \dots & \dots \\
        add & \dots & \dots & \dots & \dots & \dots & \dots & \dots & \dots & \dots & \dots \\
        cea & \dots & \dots & \dots & \dots & \dots & \dots & \dots & \dots & \dots & \dots \\
        cegar & \dots & \dots & \dots & \dots & \dots & \dots & \dots & \dots & \dots & \dots \\
        cg & \dots & \dots & \dots & \dots & \dots & \dots & \dots & \dots & \dots & \dots \\
        fastforward & \dots & \dots & \dots & \dots & \dots & \dots & \dots & \dots & \dots & \dots \\
        goalcount & \dots & \dots & \dots & \dots & \dots & \dots & \dots & \dots & \dots & \dots \\
        hm & \dots & \dots & \dots & \dots & \dots & \dots & \dots & \dots & \dots & \dots \\
        hmax & \dots & \dots & \dots & \dots & \dots & \dots & \dots & \dots & \dots & \dots \\
        landmark-cp & \dots & \dots & \dots & \dots & \dots & \dots & \dots & \dots & \dots & \dots \\
        landmark-sum & \dots & \dots & \dots & \dots & \dots & \dots & \dots & \dots & \dots & \dots \\
        merge-shrink & \dots & \dots & \dots & \dots & \dots & \dots & \dots & \dots & \dots & \dots \\
        operator-counting & \dots & \dots & \dots & \dots & \dots & \dots & \dots & \dots & \dots & \dots \\
        fF & \dots & \dots & \dots & \dots & \dots & \dots & \dots & \dots & \dots & \dots \\
        yY & \dots & \dots & \dots & \dots & \dots & \dots & \dots & \dots & \dots & \dots \\
        fFyY & \dots & \dots & \dots & \dots & \dots & \dots & \dots & \dots & \dots & \dots \\
        \bottomrule
    \end{tabular}
\end{table*}



\subsubsection{Goal Recognition}
Classical planning, as discussed in \citet{Fox2003}, treats time as relative and considers only causal dependencies among actions. However, real-world problems often entail complexities, including temporal aspects, numerical values, stochastic effects, and dynamic environments. Numeric planning extends classical planning by incorporating numeric state variables and utilising languages such as PDDL 2.1~\cite{Fox2003} and PDDL+~\cite{Fox2006}. These formalisms enable the representation of time-dependent changes, either as discrete time-dependent effects of durative actions or continuous process-dependent alterations. PDDL+ serves as an extension of PDDL designed specifically to model hybrid systems by integrating continuous processes and events, as outlined by~\citet{Haslum2019}. Its primary purpose is to facilitate the representation of planning domains that combine discrete and continuous elements.

Introduced in \citet{Scala2016} and \citet{ScalaHaslum2016}, the Expressive Numeric Heuristic Search Planner (ENHSP) is compatible with both PDDL 2.1 and PDDL+. ENHSP operates as a forward heuristic search planner, converting PDDL into an asymptotic relaxed planning graph. In this graph, nodes correspond to states explored by the planner, and a heuristic function guides the search process. This function directs the exploration towards nodes whose associated states are reachable from the initial state and bring the system closer to the desired goals.


    \section{Conclusions and Future Work}
\begin{enumerate}
    \item \textbf{Summary of the technical implementation and its contributions to adaptive learning.}
    \item \textbf{Evaluation of the framework's success in supporting diverse learning styles.}
    \item \textbf{Discussion of the practical implications of LEARNS for educational institutions.}
    \item \textbf{Identification of areas for future development and improvement.}
    \item \textbf{Final remarks on the potential impact of LEARNS in the field of educational technology.}
\end{enumerate}

    
    % What are the main contributions?

    % Do we really want to do goal recognition at this point? This is plan recognition for multiple agents (being a series of single-agent planning problems with the possibility to have them be co-operative if possible) - isn't it?

    % In goal rec, paragraph about ENHSP but don't mention specific planner until the experimentation.

    % What kind of actions can be taken from the timetabling problem? We have a weekly granularity -- we can't do anything about this at this stage. What would we require in the model to meaningfuly do this.
    
    % How can we make this slightly more complex so it is a challenge for planning/goal recognition.
    
    % Showcase the meaningful problem.
    
    % Check planning problems for PDDL -- I've contacted Eva Onaindia about this, to no avail.
    
    % what happens if a student fails? hypothesis for resit? - I get why, but wouldn't that just be common sense? Wouldn't it be advisable to assume that would be the case in the first place?
    
    % unidentified support need??? can we develop this into something that is not strictly another timetabling problems? -- this is a good idea, but we need to be careful about how we do this. An unidentified support need would imply that we know enough about the student to know that they may be at risk of possibly having one, so would it be an opinion that the student itself would have to have?

    \bibliography{aaai24}
\end{document}
